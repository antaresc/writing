% --------------
% EDITOR COMMENT
% --------------
% - Cite https://lucatrevisan.wordpress.com/2008/05/11/the-spectral-partitioning-algorithm/

Last time, we alluded that the spectrum of the Laplacian allows us to continuously measure combinatorial properties of a graph. Our goal in this note will be to make this statement precise. We will demonstrate how the second eigenvalue of the Laplacian provides a robust measurement of graph connectivity. We will introduce a notion of connectivity called graph conductance, then demonstrate how it is measured by the second eigenvalue through Cheeger's isoperimetric inequality. Our proof of Cheeger's inequality follows an argument presented by Luca Trevisan. For this note, we'll again assume that $G = (V, E)$ is a $d$-regular graph. We'll denote $L_G$ as its Laplacian, $\lambda_1 \leq \ldots \leq \lambda_n$ its eigenvalues, and $v_1, \ldots, v_n$ its corresponding set of orthonormal eigenvectors.

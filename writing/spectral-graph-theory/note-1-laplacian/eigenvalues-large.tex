\section{The Largest Eigenvalues of $L_G$}

Suppose we wish to upper-bound the largest eigenvalue of $L_G$. A convenient way of writing its Rayleigh quotient uses the identity $(a - b)^2 = 2a^2 + 2b^2 - (a + b)^2$ to derive the following.
\begin{align*}
\frac{x^\top L_G x}{x^\top x}
&= \frac{\sum_{(i, j) \in E} \big( x(i) - x(j) \big)^2 }{\sum_{i \in V} x(i)^2} \\
&= \frac{\sum_{(i, j) \in E} \big( 2 x(i)^2 + 2 x(j)^2 - (x(i) + x(j))^2 \big)}{\sum_{i \in V} x(i)^2} \\
&= \frac{2 \cdot \sum_{(i, j) \in E} \big( x(i)^2 + x(j)^2 \big)}{\sum_{i \in V} x(i)^2}
  - \frac{\sum_{(i, j) \in E} \big( x(i) + x(j) \big)^2 }{\sum_{i \in V} x(i)^2} \\
&= \frac{2d \cdot \sum_{i \in V} x(i)^2}{\sum_{i \in V} x(i)^2}
  - \frac{\sum_{(i, j) \in E} \big( x(i) + x(j) \big)^2 }{\sum_{i \in V} x(i)^2} \\
&= 2d - \frac{\sum_{(i, j) \in E} \big( x(i) + x(j) \big)^2 }{\sum_{i \in V} x(i)^2}
\end{align*}

where the fourth line comes from the assumption that $G$ is $d$-regular. Notice that $\sum_{(i, j) \in E} \big( x(i) + x(j) \big)^2$ is non-negative. The Rayleigh quotient is thus maximized when that sum is equal to 0, implying the following.

\begin{claim}
Let $\lambda_1 \leq \ldots \leq \lambda_n$ be eigenvalues of $L_G$ for $G$ a $d$-regular graph. Then $\lambda_n \leq 2d$.
\end{claim}

Now, the sum $\sum_{(i, j) \in E} \big( x(i) + x(j) \big)^2$ is 0 when we can assign $x(j) = -x(i)$  for each edge $(i, j) \in E$. However, if there exists a labeling of vertices such that for each edge, one endpoint is exactly the negative of the opposite endpoint, then the graph must be bipartite. Again this relationship is tight!

\begin{theorem}
A $d$-regular graph $G$ is bipartite if and only if $\lambda_n = 2d$
\end{theorem}
\begin{proof}
In the forward direction, suppose $G$ is a bipartite graph with bipartitions $A, B \subseteq V$. For all $i \in A$, assign $x(i) = 1$, and for all $i \in B$, assign $x(i) = -1$. We then have that $x(i) = -x(j)$ for each $(i, j) \in E$. In the other direction, if $\lambda_n = 2d$, then for $v_n$ the largest eigenvector, we have the following.
\begin{align*}
\frac{v_n^\top L_G v_n}{v_n^\top v_n} = 2d
&\qquad\Longleftrightarrow\qquad
2d - \frac{\sum_{(i, j) \in E} \big( v_n(i) + v_n(j) \big)^2 }{\sum_{i \in V} v_n(i)^2} = 2d \\
&\qquad\Longleftrightarrow\qquad
\frac{\sum_{(i, j) \in E} \big( v_n(i) + v_n(j) \big)^2 }{\sum_{i \in V} v_n(i)^2} = 0
\end{align*}

which can only be true if $v_n(i) = -v_n(j)$ for each $(i, j) \in E$. Take $A = \{ i \in V : v_n(i) > 0 \}$ and $B = \{ i \in V : v_n(i) < 0 \}$. Since $v_n \neq 0$ we have that $A, B$ partitions $V$ and all edges $e$ have one endpoint in $A$ and the other in $B$. Hence, $G$ is bipartite.
\end{proof}

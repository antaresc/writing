\documentclass{article}

\usepackage{fullpage}
\usepackage{microtype}
\usepackage{amsmath}
\usepackage{amsfonts}
\usepackage{amssymb}
\usepackage{amsthm}
\usepackage{algorithm}
\usepackage{algpseudocode}
\usepackage{color}
\usepackage{enumerate}
\usepackage{url,hyperref}
\usepackage[utf8]{inputenc}
\usepackage{bm}

\usepackage{thmtools}
\usepackage{thm-restate}

\newcommand{\etal}{{\em et al.}\ }
\newcommand{\assign}{\leftarrow}
\newcommand{\eps}{\varepsilon}
\newcommand{\opt}{\textup{\textsc{opt}}}
\newcommand{\alg}{\textup{\textsc{alg}}}
\newcommand{\pr}{\textup{Pr}}
\newcommand{\dd}{\mathrm{d}}

\DeclareMathOperator*{\argmin}{\arg\!\min}
\DeclareMathOperator*{\argmax}{\arg\!\max}
\DeclareMathOperator*{\trace}{\textup{tr}}
\DeclareMathOperator*{\rank}{\textup{rank}}
\DeclareMathOperator{\cut}{\mathrm{cut}}

\renewcommand{\thepage}{\arabic{page}}

% Theorem environments
\newtheorem{theorem}{Theorem}
\newtheorem{lemma}[theorem]{Lemma}
\newtheorem{proposition}[theorem]{Proposition}
\newtheorem{claim}[theorem]{Claim}
\newtheorem{corollary}[theorem]{Corollary}
\newtheorem{definition}[theorem]{Definition}
\newtheorem{conjecture}[theorem]{Conjecture}
\newtheorem{fact}[theorem]{Fact}

\theoremstyle{definition}
\newtheorem{example}[theorem]{Example}

\usepackage[T1]{fontenc}

\DeclareMathOperator*{\Vol}{Vol}
\DeclareMathOperator*{\E}{\mathbb{E}}
\DeclareMathOperator*{\R}{\mathbb{R}}
\DeclareMathOperator*{\Var}{Var}
\DeclareMathOperator{\erf}{erf}
\DeclareMathOperator{\dom}{dom}
\DeclareMathOperator{\sgn}{sgn}
\newcommand{\cS}{\mathcal{S}}
\newcommand{\cP}{\mathcal{P}}
\newcommand{\cH}{\mathcal{H}}
\newcommand{\cG}{\mathcal{G}}

\newcommand{\gReg}[2]{\cG_{#1, #2}^{\textup{reg}}}
\newcommand{\gErd}[2]{\cG_{#1, #2}}

\def\bone{{\bf 1}}

\allowdisplaybreaks

\setlength{\parindent}{0em}
\setlength{\parskip}{1em}

\title{Theoretical Computer Science Bibliography}
\author{Antares Chen}

\begin{document}

\maketitle


% ------------------------------------------------
% REFERENCES BY TOPIC
% ------------------------------------------------

\section{References By Topic}

\begin{enumerate}[--]

\item \emph{Brownian motion, stochastic calculus, and applications}
\begin{description}
\item[\cite{Oks13}] {{\O}}ksendal's book on It\^{o} calculus, stochastic differential equations, and various applications. Motivation for the definition of an It\^{o} Integral is pretty good.

\item[\cite{VH07}] Van Handel's lecture notes on stochastic differential equations. The first two chapters are a good recap of basic probability theory and martingales.

\item[\cite{Eld20}] Ronen Eldan's notes on stochastic calculus and applications to the KLS conjecture.

\item[\cite{EN19}] Reverse-engineers an It\^{o} process to capture random hyperplane rounding for the Goemans-Williamson maxcut relaxation.

\item[\cite{AZBG+20}] Simpler algorithm for solving maxcut (and other CSPs) using sticky Brownian motion (no slowdown compared to~\cite{EN19}).
\end{description}

% ------------------------------------------------

\item \emph{Convex optimization}
\begin{description}
\item[\cite{BNO03}] This reference was suggested by Lorenzo. Have not checked this but should.

\item[\cite{Bor10}] This reference is being used by Lorenzo for his reading group. Has content on value function formalism, Fenchel conjugacy, and other topics.

\item[\cite{Nes13}] Nesterov's textbook on convex optimization. This is like the definitive source.

\item[\cite{Nes18}] I'm not sure what the difference between what this and the one above is.
\end{description}

% ------------------------------------------------

\item \emph{Differential geometry}
\begin{description}
\item[\cite{Mil77}] This was the booked used in Math142.

\item[\cite{Tay11}] The introduction section has a good recap of multivariable calculus. The implicit function theorem gets used everywhere in differential geometry.

\item[\cite{Del16}] Harrison recommended this set of notes to learn differential forms. The PDF is in ``academic turtle shell''.

\item[\cite{Lee10}] The book everyone reads to learn about topological manifolds.

\item[\cite{Bha09}] Section 6 has discusses positive definite matrices as a manifold. Some further work is done to express the geodesics on this space.

\item[\cite{Lov19}] Lorenzo had linked this reference when discussing manifolds.

\item[\cite{RSS12}] This book develops the terminology of differential geometry and manifolds using Physics based intuition. Seems to discuss everything including differential forms, manifolds, Lie groups, symplectic geometry, and Hamiltonian mechanics.

\item[\cite{Vis18}] The introduction of this article covers the basics of differential manifolds. Later goes into applications of geodesic convexity and alternating minimization (operator scaling).

\item[\cite{BL16}] A book which introduces differential geometry of surfaces embedded in Euclidean spaces.
\end{description}

% ------------------------------------------------

% TODO: Collect some references on things like
%       multiplicative weights, follow the
%       regularized leader, gradient descent.
% \item \emph{First-order methods}
% \begin{description}
%
% \end{description}

% ------------------------------------------------

\item \emph{Graph sparsification}
% \begin{description}
%
% \end{description}

% ------------------------------------------------

\item \emph{Information geometry}
\begin{description}
\item[\cite{Nie18}] Nielsen's introduction to information geometry. Not as extensive as Amari's, also has a nice, quick introduction on differential geometry.

\item[\cite{Ama16}] Amari's book on information geometry. More in depth and formal with relevant objects and results.
\end{description}

% ------------------------------------------------

\item \emph{Large deviation theory}
\begin{description}
\item[\cite{DZ11}] Sekhar recommended me this book to study large deviation theory

\item[\cite{Var84}] It seems like this is the standard book for large deviation theory?
\end{description}

% ------------------------------------------------

\item \emph{Local algorithms and statistical physics}
\begin{description}
\item[\cite{Zad19}] Ilias Zadik's thesis. Look here for comprehensive collection of results proving overlap gap properties for various high-dimensional statistics problems.
\end{description}

% ------------------------------------------------

\item \emph{Local graph partitioning}
\begin{description}
\item[\cite{LR99}] The Leighton--Rao $O(\log n)$-approximation algorithm for sparsest cut using $\ell_1$-flow embeddings.

\item[\cite{ARV09}] The Arora--Rao--Vazirani $O(\sqrt{\log n})$-approximation for sparsest cut using $\ell_2$-flow embeddings and the expander flows framework.

\item[\cite{KRV09}] Fast sparsest cut using the cut-matching games framework to replaced having to do a multicommodity flow computation in the embedding to a single commodity flow computation.

\item[\cite{Ore11}] Lorenzo's dissertation with explanation of matrix multiplicative weights, cut matching games, and algorithms for vertex partitioning.
\end{description}

% ------------------------------------------------

\item \emph{Random graphs and statistical physics}
% \begin{description}
%
% \end{description}

% ------------------------------------------------

\item \emph{Semidefinite programming rounding, duality and complementarity}
% \begin{description}
%
% \end{description}

% ------------------------------------------------

\item \emph{Spectral graph theory}
\begin{description}
\item[\cite{RS10}] This paper introduces the Small Set Expansion Hypothesis.

\item[\cite{LRTV12}] The higher-order Cheeger inequality due to Louis, Raghavendra, Tetali, and Vempala. Theirs determines that $\phi(S_i) \leq \sqrt{\lambda_r \log r}$ (for small set $S_i$'s that partition $V$) through an argument I haven't yet parsed.

\item[\cite{LGT14}] The higher-order Cheeger inequality due to Lee, Oveis-Gharan, and Trevisan. They have a slightly improved constant, but the argument follows naturally from analyzing the heuristic of performing lower-dimensional spectral embeddings.

\item[\cite{GL13}] Improved ARV rounding for graphs with low threshold-rank.

\item[\cite{LR99}] Certifying expansion of a graph using Leighton--Rao rounding of the sparsest cut LP.

\item[\cite{ARV09}] Certifying expansion of a graph using ARV rounding of the sparsest cut SDP.

\item[\cite{AGS13}] Expander flows except applied to small sets. Basically the natural marriage of higher-order Cheeger's and expander flows.
\end{description}

% ------------------------------------------------

\item \emph{Symplectic geometry}
\begin{description}
\item[\cite{DS08}] Seems to be the definitive reference for symplectic geometry.

\item[\cite{RSS12}] This book develops discusses symplectic geometry and Hamiltonian mechanics together. Worth checking out.
\end{description}

% ------------------------------------------------

\item \emph{Random matrix theory}
\begin{description}
\item[\cite{Tro15}] Introduction to non-asymptotic matrix concentration bounds (i.e. Matrix Chernoff, etc.).
\item[\cite{BGK16}] Lecture notes on local semi-circle laws for Wigner matrices using the resolvent formalism.
\item[\cite{EY17}] Universality for random matrices using Dyson--Brownian motion as a method of interpolation.
\end{description}

% ------------------------------------------------

\item \emph{Variational calculus}
% \begin{description}
%
% \end{description}

% ------------------------------------------------

\item \emph{Vertex separators}
% \begin{description}
%
% \end{description}

\end{enumerate}


% ------------------------------------------------
% REFERENCES BY PAPER
% ------------------------------------------------

\section{References By Paper}

\begin{description}
\item[2020] \emph{Cut sparsification of the clique beyond the ramanujan bound}
\begin{enumerate}[--]
\item \cite{CR02}, \cite{DMS17}, \cite{FGL12}, \cite{Gue03}, \cite{JS17}, \cite{Neu13}, \cite{Pan14}, \cite{Par80}, \cite{Tal06}, \cite{Sen18}, \cite{SS11}, \cite{ST11}, \cite{ACK+16}, \cite{AZLO15}, \cite{BK96}, \cite{Bor19}, \cite{BSS09}, \cite{LS17}, \cite{Pol45}, \cite{ST18}, \cite{Eva10}
\end{enumerate}

\item[\cite{CHS16}] \emph{Partial resampling to approximate covering integer programs}
\begin{enumerate}[--]
\item \cite{BKNS12}, \cite{CT91}, \cite{CFLP99}, \cite{CQ18}, \cite{CQ19}, \cite{CC08}, \cite{Chv97}, \cite{Dob82}, \cite{EL73}, \cite{Fei98}, \cite{FW82}, \cite{Har16}, \cite{HS19}, \cite{Har15}, \cite{Joh73}, \cite{Kar72}, \cite{KY05}, \cite{LLRS01}, \cite{Lov75}, \cite{MT10}, \cite{PS97}, \cite{RT87}, \cite{Sla97}, \cite{Sri06}, \cite{Tre01}, \cite{Vaz13}, \cite{WRM15}, \cite{You14}
\end{enumerate}
\end{description}


% ------------------------------------------------
% DISSERTATIONS
% ------------------------------------------------

\section{Dissertations}

\begin{description}
\item[\cite{Mal19}] Enrico's dissertation with calculations of Parisi's replica symmetric and replica breaking solutions to the Sherrington--Kirkpatrick model.

\item[\cite{Ore11}] Lorenzo's dissertation with explanation of matrix multiplicative weights, cut matching games, and algorithms for vertex partitioning.

\item[\cite{Zad19}] Ilias Zadik's thesis on proving overlap gap properties to justify lower bounds for local and MCMC-based algorithms.
\end{description}


% ------------------------------------------------
% UNCATEGORIZED REFERENCES
% ------------------------------------------------

\section{Uncategorized References}

% \begin{enumerate}[--]
% \item
% \end{enumerate}


% ------------------------------------------------
% BIBLIOGRAPHY
% ------------------------------------------------

\newpage
\bibliographystyle{alpha}
\bibliography{theory.bib}

\end{document}

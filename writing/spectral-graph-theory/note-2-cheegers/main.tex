\documentclass{article}
\usepackage[utf8]{inputenc}
\usepackage{bm}
\usepackage{../../../lib/theory-scribe}
\usepackage{mathpazo}

\usepackage[T1]{fontenc}

\allowdisplaybreaks

\begin{document}

\lecture{Spectral Graph Theory: Connectivity and Cheeger's Inequality}{Antares Chen}{12/29/2019}{Fall 2019}


% --------------------------------------------------------------------
% INTRODUCTION
% --------------------------------------------------------------------
Last time, we alluded that the spectrum of the Laplacian allows us to continuously measure combinatorial properties of a graph. We will make this statement precise by introducing a notion of connectivity called graph conductance. This will lead us to Cheeger's isoperimetric inequality which relates graph conductance to the second smallest eigenvalue of the normalized Laplacian.


% Some thoughts on how to write this
% - start with two examples, the path and the clique 
% - make the kinda spurious link to connectivity
% - introduce conductance 
% - introduce the normalized laplacian
% - introduce cheeger's inequality 
% - prove cheeger's 
% - say something about the tightness of the inequality and the i n e v i t a b i l i t y of the proof
% - hint towards higher order cheegers by stating connectivity as the closeness to having two disjoint connected components and the lambda_k result 


% \bibliographystyle{plain}
% \begin{thebibliography}{99}
% \bibitem{ARV09}
% Arora, S., Rao, S., \& Vazirani, U. (2009). ``Expander flows, geometric embeddings and graph partitioning.'' In \emph{Journal of the ACM (JACM)}, 56(2), 5.
% \end{thebibliography}

\end{document}

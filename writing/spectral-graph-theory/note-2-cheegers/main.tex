\documentclass{article}

\usepackage{../../../lib/sgt-scribe}
\usepackage{../../../lib/kbordermatrix}

\renewcommand{\kbldelim}{[}
\renewcommand{\kbrdelim}{]}

\usepackage[T1]{fontenc}

\allowdisplaybreaks

\begin{document}

\lecture{Conductance and Cheeger's Inequality}{Antares Chen}{12/29/2019}{2}


% --------------------------------------------------------------------
% INTRODUCTION
% --------------------------------------------------------------------
Spectral graph theory studies combinatorial properties of graphs using eigenvalues and eigenvectors of their matrix representations. Ideas developed in this area have lead to great progress in understanding problems partitioning, flow, and routing type problems on graphs such as clustering, maxflow, and Asymmetric Traveling Salesman. This sequence of notes will cover fundamental results in spectral graph theory as well as their applications towards more recent developments in the design and analysis of graph algorithms.

Our first task will be to develop a linear algebraic representation that encodes useful combinatorial properties about graphs. We will introduce the Graph Laplacian, analyze its spectrum, then demonstrate some nice connections between eigenvalues of the Laplacian and various extremal properties of graphs.



% --------------------------------------------------------------------
% THE SECOND SMALLEST EIGENVALUE
% --------------------------------------------------------------------
\section{The Second Smallest Eigenvalue}

Towards relating graph connectivity to the second eigenvalue of the Laplacian, we'll want to look more closely at the variational characterization of $\lambda_2$. Because $\mathbb{1}$ is always a eigenvector of $L_G$ corresponding to the smallest eigenvalue, we know that the second eigenvalue must minimize the Rayleigh quotient over non-zero vectors orthogonal to $\mathbb{1}$, that is
\begin{equation}\label{eqn:second-eig}
  \lambda_2
  = \min_{x : x \neq 0, x \perp \mathbb{1}} \frac{x^\top L_G x}{x^\top x}
  = \min_{x : x \neq 0, x \perp \mathbb{1}} \frac{\sum_{(i, j) \in E} \big( x(i) - x(j) \big)^2}{\sum_{i \in V} x(i)^2}
\end{equation}

Now, let's write this in a more interpretable way by using the required condition $x \perp \mathbb{1}$. Consider the quantity $\sum_{i, j \in V} \big( x(i) - x(j) \big)^2$ i.e. the sum of differences over all pairs of vertices. We can reduce this to
\begin{align*}
\sum_{i, j \in V} \big( x(i) - x(j) \big)^2
&= \sum_{i, j \in V} \big( x(i)^2 - 2 \cdot x(i) x(j) + x(j)^2 \big) \\
&= 2n \sum_{i \in V} x(i)^2 - 2 \sum_{i, j \in V} x(i) x(j) \\
&= 2n \sum_{i \in V} x(i)^2 - 2 \bigg( \sum_{i \in V} x(i) \bigg)^2
\end{align*}

Because $x \perp \mathbb{1}$, we have $\sum_{i \in V} x(i) = 0$ implying
\begin{equation}\label{eqn:trick}
\sum_{i \in V} x(i)^2 = \frac{1}{2n} \cdot \sum_{i, j \in V} \big( x(i) - x(j) \big)^2
\end{equation}

Plugging equation~\ref{eqn:trick} into the denominator of equation~\ref{eqn:second-eig} we get
\begin{equation*}
\min_{x : x \neq 0, x \perp \mathbb{1}}
  \frac{\sum_{(i, j) \in E} \big( x(i) - x(j) \big)^2}{\sum_{i \in V} x(i)^2}
= \min_{x : x \neq 0, x \perp \mathbb{1}}
  \frac{\sum_{(i, j) \in E} \big( x(i) - x(j) \big)^2}{\frac{1}{2n} \cdot \sum_{i, j \in V} \big( x(i) - x(j) \big)^2}
\end{equation*}

This may not seem like much, but we can use a fancy choice of 1 to rewrite this into an expression that provides an interesting interpretation.
\begin{align*}
\min_{x : x \neq 0, x \perp \mathbb{1}}
  \frac{\sum_{(i, j) \in E} \big( x(i) - x(j) \big)^2}{\frac{1}{2n} \cdot \sum_{i, j \in V} \big( x(i) - x(j) \big)^2}
&= \min_{x : x \neq 0, x \perp \mathbb{1}}
  \frac{\sum_{(i, j) \in E} \big( x(i) - x(j) \big)^2}{\frac{1}{2n} \cdot \sum_{i, j \in V} \big( x(i) - x(j) \big)^2} \cdot \frac{\frac{2d}{dn}}{\frac{2}{n}} \\
&= d \cdot \Bigg( \min_{x : x \neq 0, x \perp \mathbb{1}}
  \frac{\frac{1}{dn / 2} \cdot \sum_{(i, j) \in E} \big( x(i) - x(j) \big)^2}{\frac{1}{n^2} \cdot \sum_{i, j \in V} \big( x(i) - x(j) \big)^2} \Bigg)
\end{align*}

Disregarding a factor of $d$, what this quantity says is the following. Think of the vector $x \in \mathbb{R}$ as an embedding of the vertices $i \in V$ onto the real line $\mathbb{R}$ using the map
\begin{equation*}
i \mapsto x(i)
\end{equation*}

The sum only concerns squared differences, or squared distances between vertices. In the numerator, we have the sum over all squared distances divided by $\frac{dn}{2}$ or the number of edges in $G$. This means that the numerator expresses the \emph{expected distance squared over uniformly random choices of edges from $G$}. The denominator then expresses the expected distances squared over \emph{random pairs of vertices from $G$}. If we denote $\expect_{(i, j) \sim E}$ and $\expect_{(i, j) \sim V \times V}$ as the expectation taken over random choices of edges and unordered pairs of vertices respectively, we have
\begin{equation}\label{eqn:expectation}
d \cdot \Bigg( \min_{x : x \neq 0, x \perp \mathbb{1}}
  \frac{\frac{1}{dn / 2} \cdot \sum_{(i, j) \in E} \big( x(i) - x(j) \big)^2}{\frac{1}{n^2} \cdot \sum_{i, j \in V} \big( x(i) - x(j) \big)^2} \Bigg)
= d \cdot \Bigg( \min_{x : x \neq 0, x \perp \mathbb{1}} \frac{\expect_{(i, j) \sim E} \big( x(i) - x(j) \big)^2}{\expect_{(i, j) \sim V \times V} \big( x(i) - x(j) \big)^2} \Bigg)
\end{equation}

Now to upper bound $\lambda_2 \leq \alpha$, we just need to provide an embedding $x$ that witnesses the Rayleigh quotient evaluating to $\alpha$. Choosing an embedding that minimizes this ratio thus amounts to finding an embedding of the graph on the line where the squared distance along edges is much smaller than the squared distance between a random pair of vertices. We'll now see this in action with the following example.

% --------------------------------------------------------------------
% CYCLE EXAMPLE
% --------------------------------------------------------------------

\subsection{The Cycle}

Let's upper bound $\lambda_2$ when $G$ is the $n$ vertex cycle. Our hope will be to embed the cycle onto the line such that pairs of vertices with edges between them are close to each other, while most other vertex pairs are far apart. One way to do this is as follows.

% DESCRIBE THE EMBEDDING

% MAKE SOME HAND WAVEY COMMENT ABOUT WHY THE NUMERATOR IS O(1) WHILE THE DENOMINATOR IS O(n^2)

% DO THE ACTUAL CALCULATION

% --------------------------------------------------------------------
% COMPLETE GRAPH EXAMPLE
% --------------------------------------------------------------------

\subsection{The Complete Graph}



% --------------------------------------------------------------------
% GRAPH CONDUCTANCE
% --------------------------------------------------------------------
\input{conductance.tex}


% --------------------------------------------------------------------
% CHEEGER'S INEQUALITY
% --------------------------------------------------------------------
% --------------------------------------------------------------------
% THE NORMALIZED LAPLACIAN AND CHEEGER'S INEQUALITY
% --------------------------------------------------------------------
\section{The Normalized Laplacian and Cheeger's Inequality}


% --------------------------------------------------------------------
% PROOF OF CHEEGER'S INEQUALITY
% --------------------------------------------------------------------
\section{Proof of Cheeger's Inequality}


% --------------------------------------------------------------------
% CHEEGER'S INEQUALITY IS TIGHT
% --------------------------------------------------------------------
\section{Tightness and Inevitability}



% --------------------------------------------------------------------
% CONCLUSION
% --------------------------------------------------------------------
\section{Conclusion}

In this note, we introduced the graph Laplacian as a matrix that was reverse engineered to encode cut sizes in its quadratic form. We did so for two reasons: (1) maximizing the quadratic form would provide a relaxation of the maxcut problem, and (2) doing so would hopefully allow the spectrum of the Laplacian, via the variational characterization of eigenvalues, to capture some combinatorial structure of the graph. Indeed, we verified that the spectrum of the Laplacian held a nice connection to various extremal graph properties.
\vspace{-1em}
\begin{enumerate}[-]
  \item We saw that $\lambda_2 = 0$ if the graph is disconnected.

  \item We saw that $\lambda_k = 0$ if the graph has $k$ disjoint connected components.

  \item We saw that $\lambda_n = 2d$ if a $d$-regular graph is bipartite.
\end{enumerate}

However, we have not fully discussed where relaxations come into play. Revisiting how we wrote the Rayleigh quotient for $\lambda_n$
\begin{equation*}
\frac{x^\top L_G x}{x^\top x}
= 2d - \frac{\sum_{(i, j) \in E} \big( x(i) + x(j) \big)^2 }{\sum_{i \in V} x(i)^2}
\end{equation*}

If $\lambda_n$ is large then $x(j)$ should be close to $-x(i)$ for each $(i, j) \in E$. This means that $G$ should have a large maxcut, or rather, be ``close'' to being bipartite. What this suggests is our next big idea: the eigenvalues of the Laplacian provide a robust way of continuously measuring certain combinatorial properties of a graph!


% Some thoughts on how to write this
% - start with two examples, the path and the clique
% - make the kinda spurious link to connectivity
% - introduce conductance and say how even this is interesting in its own right. One would think
%   that highly connected graphs have to be dense, but now we can ask, is there a highly connected
%   graph that is sparse? (indeed that is what expanders are).
% - Maybe cite the Jellyfish paper / and the image segmentation paper to show that it's quite useful
%   actually.
% - introduce the normalized laplacian
% - introduce cheeger's inequality: say something about why this makes spectral graph theory very useful:
%   in an algorithmic sense like it's not enough just to say that the spectrum captures
%   extremal properties, because we can always efficiently verify that by running a BFS. What makes
%   this powerful is that it really allows us to measure combinatorial properties like connectivity
%   in a continuous manner.
% - prove cheeger's
% - say something about the tightness of the inequality and the i n e v i t a b i l i t y of the proof
% - hint towards higher order cheegers by stating connectivity as the closeness to having two
%   disjoint connected components and the lambda_k result

\vfill
\begin{center}
  \begin{footnotesize}
    Last updated: \today
  \end{footnotesize}
\end{center}

\end{document}

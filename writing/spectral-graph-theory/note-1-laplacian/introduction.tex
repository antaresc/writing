Spectral graph theory studies combinatorial properties of graphs using eigenvalues and eigenvectors of their matrix representations. Ideas developed in this area have lead to great progress in understanding problems partitioning, flow, and routing type problems on graphs such as clustering, maxflow, and Asymmetric Traveling Salesman. This sequence of notes will cover fundamental results in spectral graph theory as well as their applications towards more recent developments in the design and analysis of graph algorithms.

Our first task will be to develop a linear algebraic representation that encodes useful combinatorial properties about graphs. We will introduce the Graph Laplacian, analyze its spectrum, then demonstrate some nice connections between eigenvalues of the Laplacian and various extremal properties of graphs.

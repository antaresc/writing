\documentclass{article}
\usepackage[utf8]{inputenc}
\usepackage{bm}
\usepackage{../../lib/approx-scribe}

\usepackage{todonotes}


\begin{document}
\lecture{4}{Submodular Maximization Part 2}{Antares Chen}{3/6/2019}

We conclude our discussion of maximizing submodular functions with the double greedy algorithm for unconstrained submodular maximization due to Buchbinder, Feldman, Naor, and Schwartz~\cite{BFSS15}.

\section{Unconstrained Submodular Maximization}

Previously, we provided two examples for problems that are modeled by maximizing a submodular function under cardinality constraints: maximizing float among bank accounts, and maximizing social influence. However, many more problems are captured when cardinality constraints and guaranteed monotonicity are not provided.

Recall the problem of maxcut. Given $G = (V, E)$ where each edge has weight $w_e \geq 0$, our goal is to choose a cut such that the total weight of edges crossing the cut is maximized. More precisely, we wish to find $S \subseteq V$ such that the following function is maximized.
\begin{equation}\label{eqn:maxcut}
f(S) = \sum_{e \in E(S, V - S)} w_e
\end{equation}

where $E(S, V - S) = \{ (u, v) \in E: u \in S \textup{ and } v \in V - S \}$. The objective function for maxcut is submodular.
\begin{claim}
For a graph $G = (V, E)$ with edge weights $w_e$, the function $f$ defined in equation~\ref{eqn:maxcut} is submodular.
\end{claim}
\begin{proof}
Recall that $f$ is submodular if and only if for all $S, T \subseteq V$ we have that $f(S \cup T) + f(S \cap T) \leq f(S) + f(T)$. Now, a sum of submodular functions is also submodular, thus to demonstrate that $f(S)$ is submodular, we will show that it is a sum of submodular functions. Define $f_e: 2^V \rightarrow \mathbb{R}_{\geq 0}$ for each edge $e \in E$ as
\begin{equation*}
f_e(S) = \begin{cases}
w_e & \text{if } e \in E(S, V - S) \\
0 & \text{otherwise}
\end{cases}
\end{equation*}

Suppose $e = (u, v)$ and consider any $S, T \subseteq V$. To see that $f_e$ is submodular, we verify three cases.
\vspace{-1em}
\begin{enumerate}[1.]
\item $S, T$ either contain none or both of $u, v$. In this case we have that
\begin{equation*}
f(S \cup T) + f(S \cap T)
= 0
= f(S) + f(T)
\end{equation*}

\item $S, T$ contain the same one of $u, v$. Without loss of generality, suppose $u \in S$ and $u \in T$. Observe that
\begin{equation*}
f(S \cup T) + f(S \cap T)
= f(\{ u \}) + f(\{ u \})
= 2 \cdot w_e
= f(S) + f(T)
\end{equation*}

\item $S, T$ contain one of but different $u, v$. Without loss of generality, suppose $u \in S$ and $v \in T$. Observe that
\begin{equation*}
f(S \cup T) + f(S \cap T)
= f(\{ u, v \}) + f( \varnothing )
= 0
\leq 2 \cdot w_e
= f(\{ u \}) + f(\{ v \})
= f(S) + f(T)
\end{equation*}
\end{enumerate}

In all cases, $f_e$ satisfies submodularity. Finally, because $f(S) = \sum_{e \in E} f_e(S)$, we have that $f$ is submodular.
\end{proof}

However, the maxcut objective is certainly not monotone as the edge weights may be non-zero but $f(V) = 0$.

\section{Maximizing Non-monotone Submodular Functions}

Let us precisely define the problem of maximizing a non-monotone submodular function, sometimes called \emph{unconstrained submodular maximization}. Given $E = [n]$ a ground set of elements, we wish to choose $S \subseteq E$ such that a submodular function $f: 2^E \rightarrow \mathbb{R}_{\geq 0}$ is maximized. We now present the double greedy algorithm due to Buchbinder, Feldman, Naor, and Schwartz. The algorithm provided a simple resolution to a problem that had been studied since the 1960s. We further discuss the history of this problem in the \hyperref[sec:conclusion]{\emph{Concluding Remarks}} section.

\subsection{The Double Greedy Algorithm}

The double greedy algorithm iterates through elements of $E$ deciding whether or not to add an element $i$ to a maintained solution by computing what is gained by adding $i$ or ignoring it. More precisely, if $E = [n]$ is our ground set and $f: 2^E \rightarrow \mathbb{R}_{\geq 0}$ our submodular function, then the algorithm proceeds as follows:

\noindent\fbox {
\parbox{45.5em} {
\textbf{Double Greedy}

Given ground set $E$ and submodular function $f: 2^E \rightarrow \mathbb{R}_{\geq 0}$, initialize $X_0 = \varnothing$. Do for $i = 1, \ldots, n$:

\begin{quote}
\begin{enumerate}[1.]
\item Compute $a_i$ and $r_i$ according to the following:
\begin{equation*}
a_i = f(X_{i-1} \cup \{ i \}) - f(X_{i-1})
\qquad\qquad
r_i = f(\hat{X}_{i-1} - \{ i \}) - f(\hat{X}_{i-1})
\end{equation*}

\item If $a_i \geq 0$ and $r_i < 0$, then update:
\begin{equation*}
X_i = X_{i-1} \cup \{ i \}
\end{equation*}

\item If $r_i \geq 0$ and $a_i < 0$, then update:
\begin{equation*}
X_i = X_{i-1}
\end{equation*}

\item If $a_i \geq 0$ and $r_i \geq 0$, then randomly update $X_i$ according to the following probability:
\begin{equation*}
X_i = \begin{cases}
X_{i-1} \cup \{ i \} & \text{w.p. } \frac{a_i}{a_i + r_i} \\
X_{i-1} & \text{w.p. } \frac{r_i}{a_i + r_i}
\end{cases}
\end{equation*}
\end{enumerate}
\end{quote}
}
}
\vspace{1em}

This algorithm iteratively builds a solution $X_i$ and it decides whether or not to add $i$ based on computing $a_{i}$ and $r_{i}$. These two values measure what is gained in $f$ if $i$ is either added to $X_{i-1}$ or ignored.
\vspace{-1em}
\begin{enumerate}[-]
\item If one gains more by adding $i$, i.e. $a_i \geq 0$ while $r_i < 0$, then $i$ should be added to $X_{i-1}$.

\item If one gains more by ignoring $i$, i.e. $r_i \geq 0$ while $a_i < 0$, then $i$ should be ignored.
\end{enumerate}

But what should one do when adding and ignoring $i$ both increase the value of the solution the algorithm maintains? Why not flip a biased coin. Add $i$ to $X_{i-1}$ with probability weighted by how much is gained by either action. Indeed, this is why the algorithm is called ``double greedy.'' It greedily chooses either to add or ignore $i$ depending on how much it benefits the cost of the maintained solution. If either choice yields benefit, then tiebreak randomly.

One may notice that there is a forth case. What if both actions decrease the value of the maintained solution, i.e $a_i < 0$ and $r_i < 0$. This is not possible! Doing something will always increase the cost of $X_i$. Before showing this, let us first define the set $\hat{X}_i$ as follows:
\begin{equation*}
\hat{X}_i = X_i \cup \{ i+1, \ldots, n \}
\end{equation*}

Think of $\hat{X}_i$ as the set of items that are in $X_i$ along with what could potentially be added in future iterations. It is not always the case that $\hat{X}_i \neq X_i$ always, but it does hold in general that
\begin{equation*}
\hat{X}_0 = E
\qquad\qquad\qquad
X_i \subseteq \hat{X}_i
\qquad\qquad\qquad
X_n = \hat{X}_n
\end{equation*}

Let's also think of how $\hat{X}_i$ changes based on whether step 2 or 3 is executed in the double greedy algorithm.
\vspace{-1em}
\begin{enumerate}[-]
\item If $a_i \geq 0$ and $r_i < 0$, then $\hat{X}_i = \hat{X}_{i-1}$ as $i$ is added to $X_{i-1}$

\item If $a_i < 0$ and $r_i \geq 0$, then $\hat{X}_i = \hat{X}_{i-1} - \{ i \}$ as $i$ is will never be considered for addition in the future.
\end{enumerate}

We now prove that doing anything always benefits the cost of the maintained solution.
\begin{claim}\label{claim:do-something}
$a_i + r_i \geq 0$ for all $i$.
\end{claim}
\begin{proof}
Notice that $X_{i-1} \subseteq \hat{X}_{i-1} - \{ i \}$. By submodularity, we have that
\begin{equation*}
f(\hat{X}_{i-1}) - f(\hat{X}_{i-1} - \{ i \}) \leq f(X_{i-1} \cup \{ i \}) - f(X_i)
\end{equation*}

However, $\hat{X}_{i-1} = (\hat{X}_{i-1} - \{ i \}) \cup \{ i \}$ meaning the above inequality becomes
\begin{equation*}
f((\hat{X}_{i-1} - \{ i \}) \cup \{ i \}) - f(\hat{X}_{i-1} - \{ i \}) \leq f(X_{i-1} \cup \{ i \}) - f(X_i)
\end{equation*}

The LHS of that above is $-r_i$ while the RHS is $a_i$. Thus $-r_i \leq a_i$ which is equivalent to that required.
\end{proof}

\subsection{Analysis}

Since maxcut is an $\NP$-hard problem, we cannot expect to determine the optimal set $S \subseteq E$ that maximizes a submodular $f$ efficiently unless $\ClassP = \NP$. Thus we seek to bound the approximation ratio of a solution produced from this algorithm. Let $\opt \subseteq E$ be the optimal solution and define $\opt_i$ as follows:
\begin{equation*}
\opt_i = X_i \cup (\opt \cap \{ i+1, \ldots, n \})
\end{equation*}

Consider $\opt_i$ as a sliding window of elements from $\opt$ to $X_i$ since we have $\opt_0 = \opt$ while $\opt_n = X_n$. Because the double-greedy algorithm is stochastic, our statements will hold in expectation. More precisely, we will require the following lemma.
\begin{lemma}\label{lem:big-boi}
For all $i = 1, \ldots, n$, the following holds
\begin{equation}\label{eqn:big-boi}
\expect \big[ f(\opt_{i-1}) - f(\opt_i) \big]
\leq \frac{1}{2} \cdot \expect \big[ (f(X_i) - f(X_{i-1})) + (f(\hat{X}_i) - f(\hat{X}_{i-1})) \big]
\end{equation}
\end{lemma}

We defer the proof of this statement for now and show how this implies a bound on the approximation ratio.
\begin{theorem}
The double greedy algorithm provides a $\frac{1}{2}$-approximation for non-monotone submodular maximization on expectation.
\end{theorem}
\begin{proof}
Consider the sum $\sum_{i=1}^n \expect \big[ f(\opt_{i-1}) - f(\opt_i) \big]$ and notice lemma~\ref{lem:big-boi} implies the following bound
\begin{align*}
\sum_{i=1}^n \expect \big[ f(\opt_{i-1}) - f(\opt_i) \big]
&\leq \frac{1}{2} \cdot \sum_{i=1}^n \expect \big[ (f(X_i) - f(X_{i-1})) + (f(\hat{X}_i) - f(\hat{X}_{i-1})) \big] \\
&= \frac{1}{2} \cdot \bigg( \sum_{i=1}^n \big( \expect[ f(X_i) ] - \expect[ f(X_{i-1}) ] \big) + \sum_{i=1}^n \big( \expect[ f(\hat{X}_i) ] - \expect[ f(\hat{X}_{i-1}) ] \big) \bigg)
\end{align*}

Both sums are telescoping thus we have
\begin{equation*}
\sum_{i=1}^n \expect \big[ f(\opt_{i-1}) - f(\opt_i) \big]
\leq \frac{1}{2} \cdot \big( \expect[ f(X_n) ] - \expect[ f(X_0) ] + \expect[ f(\hat{X}_n) ] - \expect[ f(\hat{X}_0) ] \big)
\end{equation*}

Notice that $f(X_0) = f(\varnothing) = 0$ while $f(\hat{X}_n) = f(X_n)$ implying $\expect[ f(X_n) ]$ is also 0. Meanwhile, $f(\hat{X}_0) = f(E)$ meaning $\expect[ f(\hat{X}_0) ]$ is a constant that is at least 0. We can conclude that
\begin{align*}
\frac{1}{2} \cdot \big( \expect[ f(X_n) ] - \expect[ f(X_0) ] + \expect[ f(\hat{X}_n) ] - \expect[ f(\hat{X}_0) ] \big)
&\leq \frac{1}{2} \cdot \big( \expect[ f(X_n) ] + \expect[ f(\hat{X}_n) ] \big) \\
&= \frac{1}{2} \cdot \big( \expect[ f(X_n) ] + \expect[ f(X_n) ] \big) \\
&= \expect[ f(X_n) ]
\end{align*}

However, $\sum_{i=1}^n \expect \big[ f(\opt_{i-1}) - f(\opt_i) \big]$ is telescoping and equal to
\begin{equation*}
\sum_{i=1}^n \expect \big[ f(\opt_{i-1}) - f(\opt_i) \big]
= \expect \big[ f(\opt_0) \big] - \expect \big[ f(\opt_n) \big]
= f(\opt) - \expect \big[ f(X_n) \big]
\end{equation*}

Thus we have
\begin{equation*}
f(\opt) - \expect \big[ f(X_n) \big]
\leq \expect[ f(X_n) ]
\qquad\qquad\Longrightarrow\qquad\qquad
\expect[ f(X_n) ] \geq \frac{1}{2} \cdot f(\opt)
\end{equation*}

as required.
\end{proof}

The final step in our analysis is to demonstrate lemma~\ref{lem:big-boi}. The proof of this lemma proceeds by checking that inequality~\ref{eqn:big-boi} holds for every conceivable case of the algorithm. Our case work splits based on whether or not $i \in \opt$, $a_i \geq 0$ and $r_i \geq 0$ for every iteration $i$. As a broad overview of the proof, notice for the cases where $a_i \geq 0$ and $r_i < 0$ as well as $a_i < 0$ and $r_i \geq 0$ the element $i$ is added / ignored deterministically, thus we need not concern ourselves with computing an expectation. Additionally, we would have that $a_i \geq r_i$ or $r_i \geq a_i$ respectively. The brunt of the proof is thus use submodularity to appropriately relate the left and right side of inequality~\ref{eqn:big-boi} to either $a_i$ or $r_i$ then use the fact that either $a_i \geq r_i$ or $r_i \geq a_i$ to conclude the inequality.

In the case where we need to flip a biased coin, we fall back to the first two cases with a certain weighted probability. Consequently, the expectation that we need to compute recycles the same computation performed in the first two cases of our analysis. We will provide a sketch of the argument for the case when $i \notin \opt$. The analysis for $i \in \opt$ is similar.

\begin{proof}[Proof sketch of lemma~\ref{lem:big-boi}]
Assume that $i \notin \opt$. We verify inequality~\ref{eqn:big-boi} for the following cases.
\begin{enumerate}
\item Suppose $a_i \geq 0$ and $r_i < 0$: element $i$ is added to the maintained solution $X_{i-1}$ deterministically. Since $i \notin \opt$, we have that
\begin{equation*}
\opt_{i-1} = X_{i-1} \cup (\opt \cap \{ i, \ldots, n \}) \subseteq (X_i \cup \{ i + 1, \ldots, n \}) - \{ i \} = \hat{X}_i - \{ i \}
\end{equation*}

thus by submodularity of $f$, we can conclude that
\begin{equation*}
f((\hat{X}_{i-1} - \{ i \}) \cup \{ i \}) - f(\hat{X}_{i-1} - \{ i \})
\leq f(\opt_{i-1} \cup \{ i \}) - f(\opt_{i-1})
\end{equation*}

Now, there are a couple things to notice about this inequality. Focusing on the LHS, we have that $\opt_{i-1} \cup \{ i \} = \opt_i$ because $i$ is added to $X_{i-1}$. Next consider the RHS. The set $(\hat{X}_{i-1} - \{ i \}) \cup \{ i \}$ is just $\hat{X}_{i-1}$ meaning the RHS is equivalent to $f(\hat{X}_{i-1}) - f(\hat{X}_{i-1} - \{ i \})$, but $f(\hat{X}_{i-1}) - f(\hat{X}_{i-1} - \{ i \}) = -r_i$ by definition. This reasoning allows us to conclude:
\begin{equation}\label{eqn:case-1-lhs}
f(\opt_{i-1}) - f(\opt_i) \leq r_i
\end{equation}

Let us now expand $\frac{1}{2} \cdot a_i$.
\begin{equation}\label{eqn:case-1-rhs}
\frac{1}{2} \cdot a_i
= \frac{1}{2} \cdot \big( f(X_i \cup \{ i \}) - f(X_{i-1}) \big)
= \frac{1}{2} \cdot \big( (f(X_i \cup \{ i \}) - f(X_{i-1})) + (f(\hat{X}_i) - f(\hat{X}_{i-1})) \big)
\end{equation}

The addition of term $f(\hat{X}_i) - f(\hat{X}_{i-1})$ follows as $f(\hat{X}_i) - f(\hat{X}_{i-1}) = 0$. This is because $\hat{X}_i = \hat{X}_{i-1}$ since $i$ is added to $X_{i-1}$. By assumption $r_i < 0$ and $a_i \geq 0$ thus $r_i \leq \frac{1}{2} \cdot a_i$. The above calculations combine to derive the required inequality.

\item Suppose $a_i < 0$ and $r_i \geq 0$: element $i$ is ignored from the maintained solution meaning $X_{i-1} = X_i$. Consequently $\opt_{i-1} = \opt_i$. We thus have that
\begin{equation}\label{case-2-lhs}
f(\opt_{i-1}) - f(\opt_i) = 0
\end{equation}

Certainly, $0 \leq \frac{1}{2} \cdot r_i$ thus expanding out $r_i$ derives
\begin{equation}\label{case-2-rhs}
\frac{1}{2} \cdot r_i
= \frac{1}{2} \cdot \big( f(\hat{X}_{i-1} \cup \{ i \}) - f(\hat{X}_{i-1}) \big)
= \frac{1}{2} \cdot \big( (f(X_i \cup \{ i \}) - f(X_{i-1})) + (f(\hat{X}_i) - f(\hat{X}_{i-1})) \big)
\end{equation}

The addition of term $f(X_i \cup \{ i \}) - f(X_{i-1})$ follows as $f(X_i \cup \{ i \}) - f(X_{i-1}) = 0$ because $X_i = X_{i-1}$. Combining our calculations for this case derives the required inequality.

\item Suppose $a_i \geq 0$ and $r_i \geq 0$: With probability $\frac{a_i}{a_i + r_i}$ we fall into case 1, and with probability $\frac{r_i}{a_i + r_i}$ into case 2. However, in case 1, we have by inequality~\ref{eqn:case-1-lhs}
\begin{equation*}
f(\opt_{i-1}) - f(\opt_i) \leq r_i
\end{equation*}

while in case 2, inequality~\ref{case-2-lhs} implies
\begin{equation*}
f(\opt_{i-1}) - f(\opt_i) \leq 0
\end{equation*}

We can thus compute the expectation $\expect \big[ f(\opt_{i-1}) - f(\opt_i) \big]$ as follows
\begin{equation}\label{eqn:expect-LHS}
\expect \big[ f(\opt_{i-1}) - f(\opt_i) \big]
\leq \frac{a_i}{a_i + r_i} \cdot r_i + \frac{r_i}{a_i + r_i} \cdot 0
= \frac{a_i r_i}{a_i + r_i}
\end{equation}

Now the RHS of the required inequality~\ref{eqn:big-boi} can be written as
\begin{equation*}
\frac{1}{2} \cdot \expect \big[ f(X_i) - f(X_{i-1}) \big] + \frac{1}{2} \cdot \expect \big[ f(\hat{X}_i) - f(\hat{X}_{i-1}) \big]
\end{equation*}

Let's look at these two expectations separately. For $\expect \big[ f(X_i) - f(X_{i-1}) \big]$, we are in case 1 with probability $\frac{a_i}{a_i + r_i}$ and by equation~\ref{eqn:case-1-rhs}
\begin{equation*}
f(X_i) - f(X_{i-1}) = a_i
\end{equation*}

With probability $\frac{r_i}{a_i + r_i}$ we are in case 2, and because $\hat{X}_{i-1} = \hat{X}_i$, we have
\begin{equation*}
f(\hat{X}_i) - f(\hat{X}_{i-1}) = 0
\end{equation*}

Consequently, we derive
\begin{equation*}
\frac{1}{2} \cdot \expect \big[ f(X_i) - f(X_{i-1}) \big]
= \frac{1}{2} \cdot \bigg( \frac{a_i}{a_i + r_i} \cdot a_i + \frac{r_i}{a_i + r_i} \cdot 0 \bigg)
= \frac{1}{2} \cdot \frac{a_i^2}{a_i + r_i}
\end{equation*}

For $\expect \big[ f(\hat{X}_i) - f(\hat{X}_{i-1}) \big]$, case 1 admits $\hat{X}_i = \hat{X}_{i-1}$ thus
\begin{equation*}
f(\hat{X}_i) - f(\hat{X}_{i-1}) = 0
\end{equation*}

while case 2 admits
\begin{equation}
f(\hat{X}_i) - f(\hat{X}_{i-1}) = r_i
\end{equation}

by equation~\ref{case-2-rhs}. The expectation is thus:
\begin{equation*}
\frac{1}{2} \cdot \expect \big[ f(X_i) - f(X_{i-1}) \big]
= \frac{1}{2} \cdot \bigg( \frac{a_i}{a_i + r_i} \cdot 0 + \frac{r_i}{a_i + r_i} \cdot r_i \bigg)
= \frac{1}{2} \cdot \frac{r_i^2}{a_i + r_i}
\end{equation*}

Combining these two expectations, we derive
\begin{equation}\label{eqn:expect-RHS}
\frac{1}{2} \cdot \expect \big[ ( f(X_i) - f(X_{i-1}) ) + ( f(\hat{X}_i) - f(\hat{X}_{i-1}) ) \big]
= \frac{1}{2} \cdot \frac{a_i^2 + r_i^2}{a_i + r_i}
\end{equation}

Now for any $a_i, r_i$ we have that
\begin{equation*}
\frac{a_i r_i}{a_i + r_i}
\leq \frac{1}{2} \cdot \frac{a_i^2 + r_i^2}{a_i + r_i}
\end{equation*}

since $(a_i - r_i)^2 \geq 0$. By equations~\ref{eqn:expect-RHS} and~\ref{eqn:expect-LHS} we can conclude that
\begin{equation*}
\expect \big[ f(\opt_{i-1}) - f(\opt_i) \big]
\leq \frac{1}{2} \cdot \expect \big[ ( f(X_i) - f(X_{i-1}) ) + ( f(\hat{X}_i) - f(\hat{X}_{i-1}) ) \big]
\end{equation*}

as required.
\end{enumerate}

By claim~\ref{claim:do-something}, we need not consider the case $a_i < 0$ and $r_i < 0$. In all cases, at least for $i \notin \opt$, the required inequality~\ref{eqn:big-boi} holds thus completing the proof sketch.
\end{proof}

\section{Concluding Remarks}\label{sec:conclusion}

The search for an optimal approximation algorithm for maximizing a non-monotone submodular function has a storied history. The study of this problem first began in the 1960s. A number of algorithms discovered during this period either focused on special cases, or provided outputs without provable guarantees~\cite{GST99, Kha68, GG05}. It wasn't until 2007, when Feige, Mirrokni and Vondr\'{a}k first provided a rigorous study of the problem~\cite{FMV07}. In their paper, they prove the following hardness result:
\begin{theorem}
For any $\epsilon > 0$, there does not exist $(\frac{1}{2} + \epsilon)$-approximation algorithm for maximizing a non-monotone submodular function using polynomial number of queries to an oracle access.
\end{theorem}

Think of an access oracle as some black-box that provides the value of the submodular function $f$ given $S \subseteq E$. Additionally, they give a number of constant factor approximation algorithms for maximizing a non-montone submodular function -- notably a 0.4-approximation using local search.

For the next five years, improvements to the approximation algorithm seemed to add complexity while only making incremental gains in the approximation ratio. Gharan and Vondr\'{a}k~\cite{GV11} first improved the algorithm to provide a 0.41-approximation using a technique called simulating annealing. Feldman, Naor, and Schwartz~\cite{FNS11} then provided a more nuanced analysis to increase the approximation ratio to 0.42. Finally in 2012, Buchbinder, Feldman, Naor, and Schwartz~\cite{BFSS15} published the 0.5-approximation discussed above, providing a remarkably simple algorithm that achieved optimality without requiring complex analysis seen in preceding work. The problem of derandomizing the double greedy algorithm was resolved only in the past year by~\cite{BF18}.

\bibliographystyle{plain}
\begin{thebibliography}{99}
\bibitem{BF18}
N. Buchbinder, \& M. Feldman. ``Deterministic algorithms for submodular maximization problems.'' In \emph{ACM Transactions on Algorithms} (2018), 14(3), 32.

\bibitem{BFSS15}
N. Buchbinder, M. Feldman, J. Seffi, \& R. Schwartz. ``A tight linear time (1/2)-approximation for unconstrained submodular maximization.'' In \emph{SIAM Journal on Computing} (2015), 44(5), 1384-1402.

\bibitem{FMV07}
U. Feige, V. Mirrokni \& J. Vondr\'{a}k. ``Maximizing non-monotone submodular functions.'' In \emph{FOCS} (2007), 461–471.

\bibitem{FNS11}
M. Feldman, J. Naor, and R. Schwartz. ``Nonmonotone submodular maximization via a structural continuous greedy algorithm.'' In \emph{ICALP} (2011), 342–353.

\bibitem{GG05}
Goldengorin, B., \& Ghosh, D. (2005). ``A multilevel search algorithm for the maximization of submodular functions applied to the quadratic cost partition problem.'' In \emph{Journal of Global Optimization}, 32(1), 65-82.

\bibitem{GST99}
Goldengorin, B., Sierksma, G., Tijssen, G. A., \& Tso, M. (1999). ``The data-correcting algorithm for the minimization of supermodular functions.'' In \emph{Management Science}, 45(11), 1539-1551.

\bibitem{GV11}
S. O. Gharan \& J. Vondr\'{a}k. ``Submodular maximization by simulated annealing.'' In \emph{SODA} (2011), 1098–1117

\bibitem{Kha68}
Khachaturov, V. R. (1968). ``Some problems of the consecutive calculation method and its applications to location problems'' (Doctoral dissertation, Ph. D. thesis, Central Economics \& Mathematics Institute, Russian Academy of Sciences, Moscow, 1968 (in Russian)).
\end{thebibliography}

\end{document}

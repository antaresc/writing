% -------------
% EDITING NOTES
% -------------
% - Write the block matrix in below
% - Write in how to assign x_1 and x_2

\section{Connections to Extremal Graph Properties}

If we consider how we demonstrated that $\lambda_1 = 0$, then we notice that our choice to set $x(u) = 1$ was arbitrary. $x(u)$ could be assigned any number so long as we guarantee that $x(u) = x(v)$ for each edge $(u, v) \in E$. However, imagine if $G$ had two separate connected components. We could hypothetically assign two vectors, $x$ and $x'$, such that both are orthogonal and yet map to 0 under the Laplacian of the graph implying the Laplacian would have 0 as an eigenvalue with multiplicity two. This intuition is actually tight in the sense of the following statement.

\begin{theorem}\label{thm:connectivity}
$G$ is connected if and only if $L_G$ has eigenvalue 0 with multiplicity 1.
\end{theorem}
\begin{proof}
First we demonstrate the forward direction: if $G$ is connected, then $L_G$ has eigenvalue 0 with multiplicity 1. Since we know that $L_G \mathbb{1} = 0$, we need to show that any eigenvector $x$ with eigenvalue 0 is in the span of $\mathbb{1}$. Suppose $x$ is an eigenvector with eigenvalue 0. Then $L_G x = 0$ implying
\begin{equation*}
0 
= x^\top L_G x
= \sum_{(i, j) \in E} \big( x(i) - x(j) \big)^2 
\end{equation*}

The left expression is a sum of squares meaning each term must be simultaneously equal to 0. Consequently, $x(i) = x(j)$ for each $(i, j) \in E$ implying $x = t\mathbb{1}$ for a $t \in \mathbb{R}$ as required. 

In the opposite direction, we demonstrate the converse of the claim: if $G$ is disconnected, then $L_G$ must have eigenvalue 0 with multiplicity greater than 1. We need to show $L_G$ has at least two linearly independent eigenvectors with eigenvalue 0. Now, if $G$ is disconnected, then it has two disjoint connected components. Let's denote $G_1$ and $G_2$ as the induced subgraphs on those connected components. These two graphs are edge disjoint hence the Laplacian decomposes in an additive manner
\begin{equation*}
L_G = L_{G_1} + L_{G_2}
\end{equation*}

Without loss of generality, let's assume that $L_G$ forms the following block matrix
\begin{equation*}
L_G
\end{equation*}

We can immediately see two ways to assign vectors $x_1$ and $x_2$ such that $L_G x_1 = 0$ and $L_G x_2 = 0$
\begin{equation*}
L_G x_1 = 
\qquad\qquad
L_G x_2 =
\end{equation*}

Since $\langle x_1, x_2 \rangle = 0$, they are orthogonal, hence linearly independent, as required.
\end{proof}

This demonstrates our first connection between the spectrum of the Laplacian and extremal graph properties: we can verify if the graph is connected by checking if its Laplacian has eigenvalue with multiplicity 1. Furthermore, the proof of theorem~\ref{thm:connectivity} can be extended in a straightforward manner to demonstrate the stronger statement

\begin{corollary}
$G$ has $k$ disjoint connected components if and only if $\lambda_k = 0$.
\end{corollary}

Let's turn our attention to the largest eigenvalue. Previously, we demonstrated that $\lambda_n \leq 2d$ for $d$-regular graphs. Is this upper-bound tight? To think of how we could assign the largest eigenvector, let's consider the Rayleigh quotient written

% INSERT SOMETHING ABOUT BIPARTITE-NESS
\documentclass{article}
\usepackage[utf8]{inputenc}
\usepackage{bm}
\usepackage{../../../lib/theory-scribe}
\usepackage{mathpazo}

\usepackage[T1]{fontenc}

\allowdisplaybreaks

\begin{document}

\lecture{Spectral Graph Theory: Connectivity and Cheeger's Inequality}{Antares Chen}{12/29/2019}{Fall 2019}


% --------------------------------------------------------------------
% INTRODUCTION
% --------------------------------------------------------------------
Spectral graph theory studies combinatorial properties of graphs using eigenvalues and eigenvectors of their matrix representations. Ideas developed in this area have lead to great progress in understanding problems partitioning, flow, and routing type problems on graphs such as clustering, maxflow, and Asymmetric Traveling Salesman. This sequence of notes will cover fundamental results in spectral graph theory as well as their applications towards more recent developments in the design and analysis of graph algorithms.

Our first task will be to develop a linear algebraic representation that encodes useful combinatorial properties about graphs. We will introduce the Graph Laplacian, analyze its spectrum, then demonstrate some nice connections between eigenvalues of the Laplacian and various extremal properties of graphs.


% Some thoughts on how to write this
% - start with two examples, the path and the clique 
% - make the kinda spurious link to connectivity
% - introduce conductance 
% - introduce the normalized laplacian
% - introduce cheeger's inequality 
% - prove cheeger's 
% - say something about the tightness of the inequality and the i n e v i t a b i l i t y of the proof
% - hint towards higher order cheegers by stating connectivity as the closeness to having two disjoint connected components and the lambda_k result 


% \bibliographystyle{plain}
% \begin{thebibliography}{99}
% \bibitem{ARV09}
% Arora, S., Rao, S., \& Vazirani, U. (2009). ``Expander flows, geometric embeddings and graph partitioning.'' In \emph{Journal of the ACM (JACM)}, 56(2), 5.
% \end{thebibliography}

\end{document}

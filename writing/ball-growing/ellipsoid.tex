% REMOVED SECTION ON THE ELLIPSOID METHOD

\subsection{The Ellipsoid Method and A Separation Oracle for Multicut}

Our LP rounding algorithm will first solve the linear program relaxation for a optimal continuous solution $x$. However, one look at relaxation~\ref{eq:primal-lp} raises an immediate issue. How can we efficiently solve the LP when there may be exponentially many paths between $s_i$ and $t_i$ in the given graph. There does exist a polynomial sized (with respect to the number of vertices and edges in the graph) LP relaxation for multicut. However, these notes will approach this problem using the \emph{Ellipsoid Method}.

The Ellipsoid Method is an optimization method that solves a linear program provided one can construct a polynomial-time \emph{separation oracle}. A separation oracle is a \emph{problem-specific} algorithm that operates as follows. Given a potential solution to the LP $x$, it returns \textsc{yes} if $x$ is feasible. If $x$ is not feasible, then it returns \textsc{no} and provides a constraint that $x$ violates.
\begin{theorem}
Given a linear program and a valid separation oracle, there exists an algorithm that does the following:
\vspace{-1em}
\begin{enumerate}[i.]
\item Solves for the optimal solution of the linear program.
\item Runs in polynomial time with respect to the number of variables in the LP as well as the bit-wise complexity of encoding any one constraint.
\end{enumerate}
\end{theorem}

Using the Ellipsoid Method to solve LP~\ref{eq:primal-lp} requires us to exhibit a polynomial-time separation oracle that distinguishes the feasibility of $x$.

\noindent\fbox {
\parbox{45.5em} {
\textbf{Separation Oracle}

Given a potential solution $x$ to linear program~\ref{eq:primal-lp}, do the following:

\begin{quote}
\begin{enumerate}[1.]
\item Check if all $x_e \geq 0$. Return \textsc{no} and $x_e$ if any $x_e < 0$.

\item Construct the graph $G' = (V, E)$ with edges weighted by $x_e$.

\item For all $i = 1, \ldots, k$, compute the shortest path $p_i$ between $s_i$ and $t_i$ on $G'$.

\item If the length of any $p_i$ is less than $1$, then return \textsc{no} and the violated constraint:
\begin{equation*}
\sum_{e \in p_i} x_e \geq 1
\end{equation*}

\item Return \textsc{yes} otherwise.
\end{enumerate}
\end{quote}
}
}
\vspace{1em}

Notice that this certainly runs in polynomial time with respect to the size of the graph. We need only perform $k$ shortest path computations and a linear pass through all $x_e$'s. All that remains is to show that this is a valid separation oracle.

\begin{claim}
The separation oracle returns \textsc{yes} if and only if the given $x$ is feasible.
\end{claim}
\begin{proof}
Suppose $x$ is feasible, then $x_e \geq 0$ for all $e \in E$ implying step (1) will not fail. Now consider any $s_i$, $t_i$ pair. Since $x$ is feasible, the following constraint holds for all $P \in \mathcal{P}_i$.
\begin{equation*}
\sum_{e \in P} x_e \geq 1
\end{equation*}

However, this sum is exactly the length of $P$ in graph $G'$. Since the length of any path $P \in \mathcal{P}_i$ is at least 1, the shortest path $p_i$ will also have distance at least 1 implying step (4) will not fail. The oracle will correctly return \textsc{yes}.

Now suppose $x$ is not feasible. Then either $x_e < 0$, for which step (1) will fail, or $\sum_{e \in P} x_e < 1$ for some $P \in \mathcal{P}_i$. If the length of $P$ is less than 1, then the length of the shortest path between $s_i$ and $t_i$ will be less than 1 as its length is at most that of $P$. It follows that step (4) will fail and the oracle will correctly return \textsc{no}.
\end{proof}

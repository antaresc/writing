% -------------
% EDITING NOTES
% -------------
% - Include a picture of the mini-Laplacian

\section{The Graph Laplacian}

We are now ready to begin with spectral graph theory. Suppose we are given an undirected, unweighted graph $G = (V, E)$. We will need a way of representing $G$ as a matrix that captures combinatorial properties of $G$. If our goal is to think about problems related to partitioning, routing, and computing flows in a graph, then it is reasonable to want information regarding graph \emph{cuts}. More precisely, we want to capture the size of a cut $S \subseteq V$ in the sense that, for $E(S, V - S)$ the set of edges crossing $S$
\begin{equation*}
E(S, V - S) = \{ (i, j) : i \in S \text{ and } j \notin S \}
\end{equation*}

the size $\lvert E(S, V - S) \rvert$ is encoded by the quadratic form of our matrix representation. This is interesting for an immediate reason. The problem of \emph{maxcut} asks to compute a cut $S$ maximizing $E(S, V - S)$. Maxcut is $\NP$-hard and so one might want a continuous relaxation of the problem to construct rounding algorithms. If the quadratic form of this hypothetical matrix encodes cut sizes, then maximizing the quadratic form would provide a continuous relaxation of maxcut. Additionally, we may hope that further interesting combinatorial properties related to cuts may arise when studying the spectrum of our hypothetical matrix.

Towards constructing this matrix, we'll endow $G$ with an arbitrary orientation. This seems like a strange step, but ultimately the direction of an edge will not matter. Supposing that $G$ has $n$ vertices, we can then define the indicator vector $b_e \mathbb{R}^n$ for each edge $e \in E$ where coordinate $i \in V$ is given by
\begin{equation*}
b_e(i) = \begin{cases}
  +1 & \text{if } e \text{ is directed toward } i \\
  -1 & \text{if } e \text{ is directed away from } i \\
  0 & \text{otherwise}
\end{cases}
\end{equation*}

That is for each $e \in E$, $b_e$ is $+1$ at the head of $e$, $-1$ at the tail of $e$, and 0 everywhere else. Now for a fixed edge $e = (u, v)$, consider the matrix $b_e b_e^\top$. It will be 0 everywhere except for
\begin{equation*}
b_e b_e^\top
= \kbordermatrix{
	& & u & & v & \\
	& \ldots & -1 & \ldots & +1 & \ldots
}
\cdot
\kbordermatrix{
	& \\
	& \vdots \\
  u & -1 \\
  & \vdots \\
  v & +1 \\
  & \vdots
}
= \kbordermatrix{
	& & u & & v & \\
  & & \vdots & & \vdots & \\
  u & \ldots & +1 & \ldots & -1 & \ldots \\
  & & \vdots & & \vdots & \\
  v & \ldots & -1 & \ldots & +1 & \ldots \\
  & & \vdots & & \vdots &
}
\end{equation*}

If we take the indicator vector for a cut $S \subseteq V$, i.e. $x_S$ such that $x_S(i) = 1$ if $i \in S$ and 0 otherwise, then notice what the quadratic form of $b_e b_e^\top$ with $x_S$ becomes.
\begin{equation*}
x_S^\top b_e b_e^\top x_S
= \langle x_S, b_e \rangle^2
= \begin{cases}
  1 & \text{if } e \in E(S, V - S) \\
  0 & \text{otherwise}
\end{cases}
\end{equation*}

That is $x_S^\top b_e b_e^\top x_S$ indicates if $e$ is cut by $S$. Summing $b_e b_e^\top$ across $e \in E$, the quadratic form becomes
\begin{equation*}
x_S^\top \bigg( \sum_{e \in E} b_e b_e^\top \bigg) x_S
= \sum_{e \in E} \langle x_S, b_e \rangle^2
= \sum_{e \in E} \mathbb{I} \{ e \in E(S, V - S) \}
= \lvert E(S, V - S) \rvert
\end{equation*}

which is exactly what we want! The matrix $\sum_{e \in E} b_e b_e^\top$ is called the \emph{Graph Laplacian} of $G$.
\begin{definition}[Graph Laplacian] Given a graph $G = (V, E)$, the Graph Laplacian $L_G$ is defined as
\begin{equation*}
L_G = \sum_{e \in E} b_e b_e^\top
\end{equation*}
\end{definition}

One way of thinking about this definition is that $L_G$ is a sum of ``mini'' Laplacians in the sense that $b_e b_e^\top$ is the Laplacian of a new graph $G[e]$ with the same vertices of $G$ but only the single edge $e$. This can be a convenient way of thinking about the Laplacian because if we decompose $G$ into an edge disjoint set of graphs $G_1, \ldots, G_k$, then the Laplacian decomposes in an additive manner
\begin{equation*}
L_G = L_{G_1} + L_{G_2} + \ldots + L_{G_k}
\end{equation*}

Additionally, note this definition is unaffected by the orientation of $G$. We can equivalently express $L_G$ as
\begin{equation*}
L_G = D - A
\end{equation*}

where $D$ is the diagonal matrix with the $(i, i)$-th entry given by $[D]_{ii} = \deg(i)$ and $A$ is the adjacency matrix of $G$. Since $D$ and $A$ are both real symmetric matrices, this implies $L_G$ is real symmetric. But more than that, $L_G$ is positive semidefinite.
\begin{claim}
$L_G$ is a positive semidefinite matrix.
\end{claim}
\begin{proof}
Consider any $x \in \mathbb{R}^n$. We can compute the quadratic form as follows
\begin{equation*}
x^\top L_G x
= \sum_{e \in E} \langle x_S, b_e \rangle^2
= \sum_{(i, j) \in E} \big( x(i) - x(j) \big)^2
\end{equation*}

A sum of squares is always non-zero hence $x^\top L_G x \geq 0$ for any vector $x$.
\end{proof}

We now know that all the eigenvalues of $L_G$ are non-negative. Let's dive a little deeper into the spectrum of the Laplacian. For the remainder of this note, we'll assume for simplicity that $G$ is a $d$-regular graph. We'll try to understand the range that contains the eigenvalues of $L_G$.


\begin{corollary}
$G$ has $k$ disjoint connected components if and only if $\lambda_k = 0$ and $\lambda_i > 0$ for all $i > k$.
\end{corollary}

Let's now turn our attention towards upper bounding largest eigenvalue.

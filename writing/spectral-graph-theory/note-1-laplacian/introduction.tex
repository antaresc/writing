Spectral graph theory studies combinatorial properties of graphs using eigenvalues and eigenvectors of their matrix representations. Ideas developed in this area have lead to great progress in understanding problems such as graph clustering, maxflow, and Asymmetric Traveling Salesman. This sequence of notes will cover fundamental results in spectral graph theory as well as their applications towards more recent developments in the design and analysis of algoriths. Some topics that we will encounter include graph expansion and Cheeger's inequality, electrical flows, sparsification, and polynomial methods. 

We begin by reviewing tools from linear algebra that will be useful towards our study of spectral graph theory. We will then introduce the Laplacian, a particular matrix representation of graphs that will be used throughout these notes. We will analyze the spectrum of the Laplacian, then demonstrate some connections between its~ spectrum and extremal properties of graphs.
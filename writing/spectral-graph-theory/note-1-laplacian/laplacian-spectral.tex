% -------------
% EDITING NOTES
% -------------
% - The calculation for the upper-bound is incorrect... double check that later

\section{Eigenvalues of $L_G$}

Let's now dive a little deeper into the spectrum of $L_G$. For the remainder of this note, we'll assume that $G$ is a $d$-regular graph for simplicity. A good question to start with is how small and large can the eigenvalues of $L_G$ be. For lower-bounding the eigenvalues, we know that the smallest eigenvalue minimizes the Rayleigh quotient over non-zero vectors. Since $L_G$ is PSD, we have $\frac{x^\top L_G x}{x^\top x} \geq 0$. Consequently, the minimum is achieved if we can exhibit a vector $x \neq 0$ where $x^\top L_G x = 0$. Let's write
\begin{equation*}
\frac{x^\top L_G x}{x^\top x}
= \frac{\sum_{(i, j) \in E} \big( x(i) - x(j) \big)^2 }{\sum_{i \in V} x(i)^2}
\end{equation*}

If we set $x(i) = 1$ for all $i \in V$, the Rayleigh quotient will be exactly 0. We have thus shown the following.
\begin{claim}\label{claim:smallest-eigenvalue}
Let $\lambda_1 \leq \ldots \leq \lambda_n$ be eigenvalues of $L_G$. Then $\lambda_1 = 0$ with eigenvector $v_1 = \mathbb{1}$ the all-ones vector.
\end{claim}

Now suppose we wish to upper-bound the largest eigenvalue $L_G$. A more convenient way of writing its Rayleigh quotient is the following.
\begin{align*}
\frac{x^\top L_G x}{x^\top x}
&= \frac{\sum_{(i, j) \in E} \big( x(i) - x(j) \big)^2 }{\sum_{i \in V} x(i)^2} \\
&= \frac{\sum_{(i, j) \in E} \big( x(i)^2 - 2 x(i) \cdot x(j) + x(j)^2 \big) }{\sum_{i \in V} x(i)^2} \\
&= \frac{\sum_{(i, j) \in E} x(i)^2 + x(j)^2}{\sum_{i \in V} x(i)^2} - \frac{\sum_{2 \cdot (i, j) \in E} x(i) \cdot x(j) }{\sum_{i \in V} x(i)^2} \\
&= \frac{2d \cdot \sum_{i \in V} x(i)^2}{\sum_{i \in V} x(i)^2} - \frac{2 \cdot \sum_{(i, j) \in E} x(i) \cdot x(j) }{\sum_{i \in V} x(i)^2} \\
&= 2d - \frac{2 \cdot \sum_{(i, j) \in E} x(i) \cdot x(j) }{\sum_{i \in V} x(i)^2}
\end{align*}

where the fourth line follows from the assumption that $G$ is $d$-regular. This implies the claim

\begin{claim}
Let $\lambda_1 \leq \ldots \leq \lambda_n$ be eigenvalues of $L_G$ for $G$ a $d$-regular graph. Then $\lambda_n \leq 2d$.
\end{claim}

To summarize, we have shown the following

\begin{claim}
For $G$ a $d$-regular graph, all eigenvalues $\lambda$ of $L_G$ are bounded by $0 \leq \lambda \leq 2d$. 
\end{claim}